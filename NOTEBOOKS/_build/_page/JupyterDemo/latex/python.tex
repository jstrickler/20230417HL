%% Generated by Sphinx.
\def\sphinxdocclass{jupyterBook}
\documentclass[letterpaper,10pt,english]{jupyterBook}
\ifdefined\pdfpxdimen
   \let\sphinxpxdimen\pdfpxdimen\else\newdimen\sphinxpxdimen
\fi \sphinxpxdimen=.75bp\relax
\ifdefined\pdfimageresolution
    \pdfimageresolution= \numexpr \dimexpr1in\relax/\sphinxpxdimen\relax
\fi
%% let collapsible pdf bookmarks panel have high depth per default
\PassOptionsToPackage{bookmarksdepth=5}{hyperref}
%% turn off hyperref patch of \index as sphinx.xdy xindy module takes care of
%% suitable \hyperpage mark-up, working around hyperref-xindy incompatibility
\PassOptionsToPackage{hyperindex=false}{hyperref}
%% memoir class requires extra handling
\makeatletter\@ifclassloaded{memoir}
{\ifdefined\memhyperindexfalse\memhyperindexfalse\fi}{}\makeatother

\PassOptionsToPackage{warn}{textcomp}

\catcode`^^^^00a0\active\protected\def^^^^00a0{\leavevmode\nobreak\ }
\usepackage{cmap}
\usepackage{fontspec}
\defaultfontfeatures[\rmfamily,\sffamily,\ttfamily]{}
\usepackage{amsmath,amssymb,amstext}
\usepackage{polyglossia}
\setmainlanguage{english}



\setmainfont{FreeSerif}[
  Extension      = .otf,
  UprightFont    = *,
  ItalicFont     = *Italic,
  BoldFont       = *Bold,
  BoldItalicFont = *BoldItalic
]
\setsansfont{FreeSans}[
  Extension      = .otf,
  UprightFont    = *,
  ItalicFont     = *Oblique,
  BoldFont       = *Bold,
  BoldItalicFont = *BoldOblique,
]
\setmonofont{FreeMono}[
  Extension      = .otf,
  UprightFont    = *,
  ItalicFont     = *Oblique,
  BoldFont       = *Bold,
  BoldItalicFont = *BoldOblique,
]



\usepackage[Bjarne]{fncychap}
\usepackage[,numfigreset=1,mathnumfig]{sphinx}

\fvset{fontsize=\small}
\usepackage{geometry}


% Include hyperref last.
\usepackage{hyperref}
% Fix anchor placement for figures with captions.
\usepackage{hypcap}% it must be loaded after hyperref.
% Set up styles of URL: it should be placed after hyperref.
\urlstyle{same}


\usepackage{sphinxmessages}



        % Start of preamble defined in sphinx-jupyterbook-latex %
         \usepackage[Latin,Greek]{ucharclasses}
        \usepackage{unicode-math}
        % fixing title of the toc
        \addto\captionsenglish{\renewcommand{\contentsname}{Contents}}
        \hypersetup{
            pdfencoding=auto,
            psdextra
        }
        % End of preamble defined in sphinx-jupyterbook-latex %
        

\title{Using Jupyter}
\date{Apr 20, 2023}
\release{}
\author{The Jupyter Book community}
\newcommand{\sphinxlogo}{\vbox{}}
\renewcommand{\releasename}{}
\makeindex
\begin{document}

\pagestyle{empty}
\sphinxmaketitle
\pagestyle{plain}
\sphinxtableofcontents
\pagestyle{normal}
\phantomsection\label{\detokenize{JupyterDemo::doc}}




\begin{DUlineblock}{0em}
\item[] \sphinxstylestrong{\Large Markdown cells}
\end{DUlineblock}

\sphinxAtStartPar
This is a Markdown (documentation) cell, for adding documentation and comments to a notebook. It uses Markdown, a text styling language, as well as HTML. When a Markdown cell is “run”, it converts the contents of the cell into HTML, which is then rendered and displayed.

\sphinxAtStartPar
Double\sphinxhyphen{}click on a Markdown cell to see its source.

\sphinxAtStartPar
This is \sphinxstylestrong{more} \sphinxstyleemphasis{markdown}. Surround text with double asterisk for bold, or single underline for italic.

\sphinxAtStartPar
See the Notebook named \sphinxstylestrong{JupyterMarkdownGuide} for details.

\begin{DUlineblock}{0em}
\item[] \sphinxstylestrong{\large Code Cells}
\end{DUlineblock}

\sphinxAtStartPar
Code cells contain Python (or other language) code. When a code cell is run, it executes the code, and shows the output below the cell. The output is retained unless you explicitly clear it.

\begin{sphinxuseclass}{cell}\begin{sphinxVerbatimInput}

\begin{sphinxuseclass}{cell_input}
\begin{sphinxVerbatim}[commandchars=\\\{\}]
\PYG{n}{x} \PYG{o}{=} \PYG{l+m+mi}{3}
\end{sphinxVerbatim}

\end{sphinxuseclass}\end{sphinxVerbatimInput}

\end{sphinxuseclass}
\begin{sphinxuseclass}{cell}\begin{sphinxVerbatimInput}

\begin{sphinxuseclass}{cell_input}
\begin{sphinxVerbatim}[commandchars=\\\{\}]
\PYG{k}{for} \PYG{n}{i} \PYG{o+ow}{in} \PYG{n+nb}{range}\PYG{p}{(}\PYG{l+m+mi}{3}\PYG{p}{)}\PYG{p}{:}
    \PYG{n+nb}{print}\PYG{p}{(}\PYG{n}{i}\PYG{p}{)}
\end{sphinxVerbatim}

\end{sphinxuseclass}\end{sphinxVerbatimInput}
\begin{sphinxVerbatimOutput}

\begin{sphinxuseclass}{cell_output}
\begin{sphinxVerbatim}[commandchars=\\\{\}]
0
1
2
\end{sphinxVerbatim}

\end{sphinxuseclass}\end{sphinxVerbatimOutput}

\end{sphinxuseclass}
\begin{sphinxuseclass}{cell}\begin{sphinxVerbatimInput}

\begin{sphinxuseclass}{cell_input}
\begin{sphinxVerbatim}[commandchars=\\\{\}]
\PYG{n+nb}{print}\PYG{p}{(}\PYG{l+s+s2}{\PYGZdq{}}\PYG{l+s+s2}{wombat}\PYG{l+s+s2}{\PYGZdq{}}\PYG{p}{)}
\end{sphinxVerbatim}

\end{sphinxuseclass}\end{sphinxVerbatimInput}
\begin{sphinxVerbatimOutput}

\begin{sphinxuseclass}{cell_output}
\begin{sphinxVerbatim}[commandchars=\\\{\}]
wombat
\end{sphinxVerbatim}

\end{sphinxuseclass}\end{sphinxVerbatimOutput}

\end{sphinxuseclass}
\begin{sphinxuseclass}{cell}\begin{sphinxVerbatimInput}

\begin{sphinxuseclass}{cell_input}
\begin{sphinxVerbatim}[commandchars=\\\{\}]
\PYG{n+nb}{print}\PYG{p}{(}\PYG{n}{x}\PYG{p}{)}
\end{sphinxVerbatim}

\end{sphinxuseclass}\end{sphinxVerbatimInput}
\begin{sphinxVerbatimOutput}

\begin{sphinxuseclass}{cell_output}
\begin{sphinxVerbatim}[commandchars=\\\{\}]
3
\end{sphinxVerbatim}

\end{sphinxuseclass}\end{sphinxVerbatimOutput}

\end{sphinxuseclass}
\begin{DUlineblock}{0em}
\item[] \sphinxstylestrong{\large It’s all one program}
\end{DUlineblock}

\sphinxAtStartPar
All code is run in the same instance of the Python interpreter, so that objects created in one cell are available to other cells, as long as the first cell has been run.

\begin{sphinxuseclass}{cell}\begin{sphinxVerbatimInput}

\begin{sphinxuseclass}{cell_input}
\begin{sphinxVerbatim}[commandchars=\\\{\}]
\PYG{k+kn}{from} \PYG{n+nn}{datetime} \PYG{k+kn}{import} \PYG{n}{date} \PYG{k}{as} \PYG{n}{Date}
\end{sphinxVerbatim}

\end{sphinxuseclass}\end{sphinxVerbatimInput}

\end{sphinxuseclass}
\begin{sphinxuseclass}{cell}\begin{sphinxVerbatimInput}

\begin{sphinxuseclass}{cell_input}
\begin{sphinxVerbatim}[commandchars=\\\{\}]
\PYG{n}{then} \PYG{o}{=} \PYG{n}{Date}\PYG{p}{(}\PYG{l+m+mi}{2011}\PYG{p}{,}\PYG{l+m+mi}{5}\PYG{p}{,}\PYG{l+m+mi}{22}\PYG{p}{)}
\end{sphinxVerbatim}

\end{sphinxuseclass}\end{sphinxVerbatimInput}

\end{sphinxuseclass}
\begin{sphinxuseclass}{cell}\begin{sphinxVerbatimInput}

\begin{sphinxuseclass}{cell_input}
\begin{sphinxVerbatim}[commandchars=\\\{\}]
\PYG{n+nb}{print}\PYG{p}{(}\PYG{n}{then}\PYG{o}{.}\PYG{n}{year}\PYG{p}{)}
\end{sphinxVerbatim}

\end{sphinxuseclass}\end{sphinxVerbatimInput}
\begin{sphinxVerbatimOutput}

\begin{sphinxuseclass}{cell_output}
\begin{sphinxVerbatim}[commandchars=\\\{\}]
2011
\end{sphinxVerbatim}

\end{sphinxuseclass}\end{sphinxVerbatimOutput}

\end{sphinxuseclass}
\begin{sphinxuseclass}{cell}\begin{sphinxVerbatimInput}

\begin{sphinxuseclass}{cell_input}
\begin{sphinxVerbatim}[commandchars=\\\{\}]
\PYG{o}{\PYGZpc{}}\PYG{k}{pinfo} then
\end{sphinxVerbatim}

\end{sphinxuseclass}\end{sphinxVerbatimInput}

\end{sphinxuseclass}
\begin{DUlineblock}{0em}
\item[] \sphinxstylestrong{\large Getting help}
\end{DUlineblock}

\sphinxAtStartPar
Putting a ? before (or after) any object displays help for that object. Using ?? will add more detailed help, if available. (The output will be in a separate pane at the bottom of the browser window).

\begin{sphinxuseclass}{cell}\begin{sphinxVerbatimInput}

\begin{sphinxuseclass}{cell_input}
\begin{sphinxVerbatim}[commandchars=\\\{\}]
i\PYG{o}{?}
\end{sphinxVerbatim}

\end{sphinxuseclass}\end{sphinxVerbatimInput}

\end{sphinxuseclass}
\begin{sphinxuseclass}{cell}\begin{sphinxVerbatimInput}

\begin{sphinxuseclass}{cell_input}
\begin{sphinxVerbatim}[commandchars=\\\{\}]
Date\PYG{o}{??}
\end{sphinxVerbatim}

\end{sphinxuseclass}\end{sphinxVerbatimInput}

\end{sphinxuseclass}
\begin{DUlineblock}{0em}
\item[] \sphinxstylestrong{\large Using Python’s scientific libraries}
\end{DUlineblock}

\sphinxAtStartPar
For typical use of Python’s scientific libraries, put the following at the top of the notebook in a code cell:



\sphinxAtStartPar
Other  modules and packages should be included as needed.

\begin{sphinxuseclass}{cell}\begin{sphinxVerbatimInput}

\begin{sphinxuseclass}{cell_input}
\begin{sphinxVerbatim}[commandchars=\\\{\}]
\PYG{k+kn}{import} \PYG{n+nn}{pandas} \PYG{k}{as} \PYG{n+nn}{pd}
\PYG{k+kn}{import} \PYG{n+nn}{scipy} \PYG{k}{as} \PYG{n+nn}{sp}
\PYG{k+kn}{import} \PYG{n+nn}{numpy} \PYG{k}{as} \PYG{n+nn}{np}
\PYG{k+kn}{import} \PYG{n+nn}{matplotlib}\PYG{n+nn}{.}\PYG{n+nn}{pyplot} \PYG{k}{as} \PYG{n+nn}{plt}
\PYG{k+kn}{import} \PYG{n+nn}{matplotlib} \PYG{k}{as} \PYG{n+nn}{mpl}
\PYG{k+kn}{import} \PYG{n+nn}{seaborn} \PYG{k}{as} \PYG{n+nn}{sns}
\end{sphinxVerbatim}

\end{sphinxuseclass}\end{sphinxVerbatimInput}

\end{sphinxuseclass}
\begin{DUlineblock}{0em}
\item[] \sphinxstylestrong{\large Inline plotting}
\end{DUlineblock}

\sphinxAtStartPar
After matplotlib is imported, use the \sphinxstylestrong{\%matplotlib inline} magic to display figures as part of the notebook. Otherwise, they are displayed in popup windows.

\begin{sphinxuseclass}{cell}\begin{sphinxVerbatimInput}

\begin{sphinxuseclass}{cell_input}
\begin{sphinxVerbatim}[commandchars=\\\{\}]
\PYG{o}{\PYGZpc{}}\PYG{k}{matplotlib} inline
\PYG{n}{values} \PYG{o}{=} \PYG{n}{np}\PYG{o}{.}\PYG{n}{random}\PYG{o}{.}\PYG{n}{randint}\PYG{p}{(}\PYG{l+m+mi}{1}\PYG{p}{,} \PYG{l+m+mi}{100}\PYG{p}{,} \PYG{l+m+mi}{50}\PYG{p}{)}
\PYG{n}{plt}\PYG{o}{.}\PYG{n}{plot}\PYG{p}{(}\PYG{n}{values}\PYG{p}{)}
\end{sphinxVerbatim}

\end{sphinxuseclass}\end{sphinxVerbatimInput}
\begin{sphinxVerbatimOutput}

\begin{sphinxuseclass}{cell_output}
\begin{sphinxVerbatim}[commandchars=\\\{\}]
[\PYGZlt{}matplotlib.lines.Line2D at 0x7fec61258790\PYGZgt{}]
\end{sphinxVerbatim}

\noindent\sphinxincludegraphics{{JupyterDemo_20_1}.png}

\end{sphinxuseclass}\end{sphinxVerbatimOutput}

\end{sphinxuseclass}
\begin{DUlineblock}{0em}
\item[] \sphinxstylestrong{\large Using HTML}
\end{DUlineblock}

\sphinxAtStartPar
Since Markdown is converted to HTML, any actual HTML in a Markdown cell is used.



\begin{DUlineblock}{0em}
\item[] \sphinxstylestrong{\large Magics}
\end{DUlineblock}

\sphinxAtStartPar
iPython and Jupyter notebooks have \sphinxstyleemphasis{line magics}, which are line\sphinxhyphen{}oriented. Many of them execute commands or turn iPython settings on and off.

\sphinxAtStartPar
Jupyter has \sphinxstyleemphasis{cell magics}, which apply to the entire cell.

\begin{sphinxuseclass}{cell}\begin{sphinxVerbatimInput}

\begin{sphinxuseclass}{cell_input}
\begin{sphinxVerbatim}[commandchars=\\\{\}]
\PYG{o}{\PYGZpc{}}\PYG{k}{lsmagic}
\end{sphinxVerbatim}

\end{sphinxuseclass}\end{sphinxVerbatimInput}
\begin{sphinxVerbatimOutput}

\begin{sphinxuseclass}{cell_output}
\begin{sphinxVerbatim}[commandchars=\\\{\}]
Available line magics:
\PYGZpc{}alias  \PYGZpc{}alias\PYGZus{}magic  \PYGZpc{}autoawait  \PYGZpc{}autocall  \PYGZpc{}automagic  \PYGZpc{}autosave  \PYGZpc{}bookmark  \PYGZpc{}cat  \PYGZpc{}cd  \PYGZpc{}clear  \PYGZpc{}colors  \PYGZpc{}conda  \PYGZpc{}config  \PYGZpc{}connect\PYGZus{}info  \PYGZpc{}cp  \PYGZpc{}debug  \PYGZpc{}dhist  \PYGZpc{}dirs  \PYGZpc{}doctest\PYGZus{}mode  \PYGZpc{}ed  \PYGZpc{}edit  \PYGZpc{}env  \PYGZpc{}gui  \PYGZpc{}hist  \PYGZpc{}history  \PYGZpc{}killbgscripts  \PYGZpc{}ldir  \PYGZpc{}less  \PYGZpc{}lf  \PYGZpc{}lk  \PYGZpc{}ll  \PYGZpc{}load  \PYGZpc{}load\PYGZus{}ext  \PYGZpc{}loadpy  \PYGZpc{}logoff  \PYGZpc{}logon  \PYGZpc{}logstart  \PYGZpc{}logstate  \PYGZpc{}logstop  \PYGZpc{}ls  \PYGZpc{}lsmagic  \PYGZpc{}lx  \PYGZpc{}macro  \PYGZpc{}magic  \PYGZpc{}man  \PYGZpc{}matplotlib  \PYGZpc{}mkdir  \PYGZpc{}more  \PYGZpc{}mv  \PYGZpc{}notebook  \PYGZpc{}page  \PYGZpc{}pastebin  \PYGZpc{}pdb  \PYGZpc{}pdef  \PYGZpc{}pdoc  \PYGZpc{}pfile  \PYGZpc{}pinfo  \PYGZpc{}pinfo2  \PYGZpc{}pip  \PYGZpc{}popd  \PYGZpc{}pprint  \PYGZpc{}precision  \PYGZpc{}prun  \PYGZpc{}psearch  \PYGZpc{}psource  \PYGZpc{}pushd  \PYGZpc{}pwd  \PYGZpc{}pycat  \PYGZpc{}pylab  \PYGZpc{}qtconsole  \PYGZpc{}quickref  \PYGZpc{}recall  \PYGZpc{}rehashx  \PYGZpc{}reload\PYGZus{}ext  \PYGZpc{}rep  \PYGZpc{}rerun  \PYGZpc{}reset  \PYGZpc{}reset\PYGZus{}selective  \PYGZpc{}rm  \PYGZpc{}rmdir  \PYGZpc{}run  \PYGZpc{}save  \PYGZpc{}sc  \PYGZpc{}set\PYGZus{}env  \PYGZpc{}store  \PYGZpc{}sx  \PYGZpc{}system  \PYGZpc{}tb  \PYGZpc{}time  \PYGZpc{}timeit  \PYGZpc{}unalias  \PYGZpc{}unload\PYGZus{}ext  \PYGZpc{}who  \PYGZpc{}who\PYGZus{}ls  \PYGZpc{}whos  \PYGZpc{}xdel  \PYGZpc{}xmode

Available cell magics:
\PYGZpc{}\PYGZpc{}!  \PYGZpc{}\PYGZpc{}HTML  \PYGZpc{}\PYGZpc{}SVG  \PYGZpc{}\PYGZpc{}bash  \PYGZpc{}\PYGZpc{}capture  \PYGZpc{}\PYGZpc{}debug  \PYGZpc{}\PYGZpc{}file  \PYGZpc{}\PYGZpc{}html  \PYGZpc{}\PYGZpc{}javascript  \PYGZpc{}\PYGZpc{}js  \PYGZpc{}\PYGZpc{}latex  \PYGZpc{}\PYGZpc{}markdown  \PYGZpc{}\PYGZpc{}perl  \PYGZpc{}\PYGZpc{}prun  \PYGZpc{}\PYGZpc{}pypy  \PYGZpc{}\PYGZpc{}python  \PYGZpc{}\PYGZpc{}python2  \PYGZpc{}\PYGZpc{}python3  \PYGZpc{}\PYGZpc{}ruby  \PYGZpc{}\PYGZpc{}script  \PYGZpc{}\PYGZpc{}sh  \PYGZpc{}\PYGZpc{}svg  \PYGZpc{}\PYGZpc{}sx  \PYGZpc{}\PYGZpc{}system  \PYGZpc{}\PYGZpc{}time  \PYGZpc{}\PYGZpc{}timeit  \PYGZpc{}\PYGZpc{}writefile

Automagic is ON, \PYGZpc{} prefix IS NOT needed for line magics.
\end{sphinxVerbatim}

\end{sphinxuseclass}\end{sphinxVerbatimOutput}

\end{sphinxuseclass}
\begin{sphinxuseclass}{cell}\begin{sphinxVerbatimInput}

\begin{sphinxuseclass}{cell_input}
\begin{sphinxVerbatim}[commandchars=\\\{\}]
\PYG{o}{!}hostname
\PYG{n}{h} \PYG{o}{=} \PYG{o}{!}hostname
\PYG{n+nb}{print}\PYG{p}{(}\PYG{n}{h}\PYG{p}{)}
\end{sphinxVerbatim}

\end{sphinxuseclass}\end{sphinxVerbatimInput}
\begin{sphinxVerbatimOutput}

\begin{sphinxuseclass}{cell_output}
\begin{sphinxVerbatim}[commandchars=\\\{\}]
Johns\PYGZhy{}Macbook.attlocal.net
\end{sphinxVerbatim}

\begin{sphinxVerbatim}[commandchars=\\\{\}]
[\PYGZsq{}Johns\PYGZhy{}Macbook.attlocal.net\PYGZsq{}]
\end{sphinxVerbatim}

\end{sphinxuseclass}\end{sphinxVerbatimOutput}

\end{sphinxuseclass}
\begin{DUlineblock}{0em}
\item[] \sphinxstylestrong{\large Running external Python scripts}
\end{DUlineblock}

\sphinxAtStartPar
Use the \sphinxstylestrong{\%run} magic to launch an external Python script

\begin{sphinxuseclass}{cell}\begin{sphinxVerbatimInput}

\begin{sphinxuseclass}{cell_input}
\begin{sphinxVerbatim}[commandchars=\\\{\}]
\PYG{o}{\PYGZpc{}}\PYG{k}{run} ../EXAMPLES/my\PYGZus{}vars.py
\end{sphinxVerbatim}

\end{sphinxuseclass}\end{sphinxVerbatimInput}
\begin{sphinxVerbatimOutput}

\begin{sphinxuseclass}{cell_output}
\begin{sphinxVerbatim}[commandchars=\\\{\}]
\PYGZlt{}Figure size 432x288 with 0 Axes\PYGZgt{}
\end{sphinxVerbatim}

\end{sphinxuseclass}\end{sphinxVerbatimOutput}

\end{sphinxuseclass}
\begin{sphinxuseclass}{cell}\begin{sphinxVerbatimInput}

\begin{sphinxuseclass}{cell_input}
\begin{sphinxVerbatim}[commandchars=\\\{\}]
\PYG{n+nb}{print}\PYG{p}{(}\PYG{n}{user\PYGZus{}name}\PYG{p}{)}
\PYG{n+nb}{print}\PYG{p}{(}\PYG{n}{animal}\PYG{p}{)}
\PYG{n+nb}{print}\PYG{p}{(}\PYG{n}{snake}\PYG{p}{)}
\end{sphinxVerbatim}

\end{sphinxuseclass}\end{sphinxVerbatimInput}
\begin{sphinxVerbatimOutput}

\begin{sphinxuseclass}{cell_output}
\begin{sphinxVerbatim}[commandchars=\\\{\}]
Susan
wombat
Eastern Racer
\end{sphinxVerbatim}

\end{sphinxuseclass}\end{sphinxVerbatimOutput}

\end{sphinxuseclass}
\begin{DUlineblock}{0em}
\item[] \sphinxstylestrong{\large Loading scripts into cells}
\end{DUlineblock}

\sphinxAtStartPar
Use the \sphinxstylestrong{\%load} magic to read a separate Python script into the current cell. After it’s loaded, it can be run like any other cell.

\sphinxAtStartPar
Once the code is loaded into the cell, the \sphinxstylestrong{\%load} command is commented out

\begin{sphinxuseclass}{cell}\begin{sphinxVerbatimInput}

\begin{sphinxuseclass}{cell_input}
\begin{sphinxVerbatim}[commandchars=\\\{\}]
\PYG{c+c1}{\PYGZsh{} \PYGZpc{}load ../EXAMPLES/read\PYGZus{}tyger.py}

\PYG{k}{with} \PYG{n+nb}{open}\PYG{p}{(}\PYG{l+s+s2}{\PYGZdq{}}\PYG{l+s+s2}{../DATA/tyger.txt}\PYG{l+s+s2}{\PYGZdq{}}\PYG{p}{,} \PYG{l+s+s2}{\PYGZdq{}}\PYG{l+s+s2}{r}\PYG{l+s+s2}{\PYGZdq{}}\PYG{p}{)} \PYG{k}{as} \PYG{n}{tyger\PYGZus{}in}\PYG{p}{:}  \PYG{c+c1}{\PYGZsh{} tyger\PYGZus{}in is return value of open()}
    \PYG{k}{for} \PYG{n}{raw\PYGZus{}line} \PYG{o+ow}{in} \PYG{n}{tyger\PYGZus{}in}\PYG{p}{:}  \PYG{c+c1}{\PYGZsh{} tyger\PYGZus{}in is iterable of lines in the file}
        \PYG{n}{line} \PYG{o}{=} \PYG{n}{raw\PYGZus{}line}\PYG{o}{.}\PYG{n}{rstrip}\PYG{p}{(}\PYG{p}{)}  \PYG{c+c1}{\PYGZsh{} remove \PYGZbs{}n}
        \PYG{n+nb}{print}\PYG{p}{(}\PYG{n}{line}\PYG{p}{)}
\end{sphinxVerbatim}

\end{sphinxuseclass}\end{sphinxVerbatimInput}
\begin{sphinxVerbatimOutput}

\begin{sphinxuseclass}{cell_output}
\begin{sphinxVerbatim}[commandchars=\\\{\}]
           The Tyger

Tyger! Tyger! burning bright
In the forests of the night,
What immortal hand or eye
Could frame thy fearful symmetry?

In what distant deeps or skies
Burnt the fire of thine eyes?
On what wings dare he aspire?
What the hand dare seize the fire?

And what shoulder, \PYGZam{} what art,
Could twist the sinews of thy heart?
And when thy heart began to beat,
What dread hand? \PYGZam{} what dread feet?

What the hammer? what the chain?
In what furnace was thy brain?
What the anvil? what dread grasp
Dare its deadly terrors clasp?

When the stars threw down their spears
And water\PYGZsq{}d heaven with their tears,
Did he smile his work to see?
Did he who made the Lamb make thee?

Tyger! Tyger! burning bright
In the forests of the night,
What immortal hand or eye
Dare frame thy fearful symmetry?

                    by William Blake
\end{sphinxVerbatim}

\end{sphinxuseclass}\end{sphinxVerbatimOutput}

\end{sphinxuseclass}
\begin{sphinxuseclass}{cell}\begin{sphinxVerbatimInput}

\begin{sphinxuseclass}{cell_input}
\begin{sphinxVerbatim}[commandchars=\\\{\}]
\PYG{o}{\PYGZpc{}}\PYG{k}{load} imports.py
\end{sphinxVerbatim}

\end{sphinxuseclass}\end{sphinxVerbatimInput}

\end{sphinxuseclass}
\begin{DUlineblock}{0em}
\item[] \sphinxstylestrong{\large Using LaTeX}
\end{DUlineblock}

\sphinxAtStartPar
Markdown cells can render LaTeX via MathJax. Put the LaTeX code inside a pair of dollar signs: \sphinxstylestrong{\$\textbackslash{}rho\$:}

\sphinxAtStartPar
\(\rho\), \(\rho\), \(\rho\) your boat

\sphinxAtStartPar
\(\mathbf{V}_1 \times \mathbf{V}_2 =  \begin{vmatrix} \
\mathbf{i} & \mathbf{j} & \mathbf{k}  \\
\frac{\partial X}{\partial u} & \frac{\partial Y}{\partial u} & 0\\
\frac{\partial X}{\partial v} & \frac{\partial Y}{\partial v} & 0\\
\end{vmatrix}\)

\sphinxAtStartPar
\(\left( \sum_{k=1}^n a_k b_k \right)^2 \leq \left( \sum_{k=1}^n a_k^2 \right) \left( \sum_{k=1}^n b_k^2 \right)\)

\begin{DUlineblock}{0em}
\item[] \sphinxstylestrong{\large Can you read this limerick?}
\end{DUlineblock}

\sphinxAtStartPar
\(\frac{12 + 144  + 20 + 3\sqrt{4}}{7} + (5 * 11) = 9^2 + 0\) 

\sphinxAtStartPar
See text of limerick at the bottom of this notebook

\begin{DUlineblock}{0em}
\item[] \sphinxstylestrong{\large Getting info}
\end{DUlineblock}

\begin{sphinxuseclass}{cell}\begin{sphinxVerbatimInput}

\begin{sphinxuseclass}{cell_input}
\begin{sphinxVerbatim}[commandchars=\\\{\}]
\PYG{k+kn}{from} \PYG{n+nn}{scipy} \PYG{k+kn}{import} \PYG{n}{stats}
\PYG{n}{sp}\PYG{o}{.}\PYG{n}{info}\PYG{p}{(}\PYG{n}{stats}\PYG{p}{)}
\end{sphinxVerbatim}

\end{sphinxuseclass}\end{sphinxVerbatimInput}
\begin{sphinxVerbatimOutput}

\begin{sphinxuseclass}{cell_output}
\begin{sphinxVerbatim}[commandchars=\\\{\}]
.. \PYGZus{}statsrefmanual:

==========================================
Statistical functions (:mod:`scipy.stats`)
==========================================

.. currentmodule:: scipy.stats

This module contains a large number of probability distributions,
summary and frequency statistics, correlation functions and statistical
tests, masked statistics, kernel density estimation, quasi\PYGZhy{}Monte Carlo
functionality, and more.

Statistics is a very large area, and there are topics that are out of scope
for SciPy and are covered by other packages. Some of the most important ones
are:

\PYGZhy{} `statsmodels \PYGZlt{}https://www.statsmodels.org/stable/index.html\PYGZgt{}`\PYGZus{}\PYGZus{}:
  regression, linear models, time series analysis, extensions to topics
  also covered by ``scipy.stats``.
\PYGZhy{} `Pandas \PYGZlt{}https://pandas.pydata.org/\PYGZgt{}`\PYGZus{}\PYGZus{}: tabular data, time series
  functionality, interfaces to other statistical languages.
\PYGZhy{} `PyMC \PYGZlt{}https://docs.pymc.io/\PYGZgt{}`\PYGZus{}\PYGZus{}: Bayesian statistical
  modeling, probabilistic machine learning.
\PYGZhy{} `scikit\PYGZhy{}learn \PYGZlt{}https://scikit\PYGZhy{}learn.org/\PYGZgt{}`\PYGZus{}\PYGZus{}: classification, regression,
  model selection.
\PYGZhy{} `Seaborn \PYGZlt{}https://seaborn.pydata.org/\PYGZgt{}`\PYGZus{}\PYGZus{}: statistical data visualization.
\PYGZhy{} `rpy2 \PYGZlt{}https://rpy2.github.io/\PYGZgt{}`\PYGZus{}\PYGZus{}: Python to R bridge.


Probability distributions
=========================

Each univariate distribution is an instance of a subclass of `rv\PYGZus{}continuous`
(`rv\PYGZus{}discrete` for discrete distributions):

.. autosummary::
   :toctree: generated/

   rv\PYGZus{}continuous
   rv\PYGZus{}discrete
   rv\PYGZus{}histogram

Continuous distributions
\PYGZhy{}\PYGZhy{}\PYGZhy{}\PYGZhy{}\PYGZhy{}\PYGZhy{}\PYGZhy{}\PYGZhy{}\PYGZhy{}\PYGZhy{}\PYGZhy{}\PYGZhy{}\PYGZhy{}\PYGZhy{}\PYGZhy{}\PYGZhy{}\PYGZhy{}\PYGZhy{}\PYGZhy{}\PYGZhy{}\PYGZhy{}\PYGZhy{}\PYGZhy{}\PYGZhy{}

.. autosummary::
   :toctree: generated/

   alpha             \PYGZhy{}\PYGZhy{} Alpha
   anglit            \PYGZhy{}\PYGZhy{} Anglit
   arcsine           \PYGZhy{}\PYGZhy{} Arcsine
   argus             \PYGZhy{}\PYGZhy{} Argus
   beta              \PYGZhy{}\PYGZhy{} Beta
   betaprime         \PYGZhy{}\PYGZhy{} Beta Prime
   bradford          \PYGZhy{}\PYGZhy{} Bradford
   burr              \PYGZhy{}\PYGZhy{} Burr (Type III)
   burr12            \PYGZhy{}\PYGZhy{} Burr (Type XII)
   cauchy            \PYGZhy{}\PYGZhy{} Cauchy
   chi               \PYGZhy{}\PYGZhy{} Chi
   chi2              \PYGZhy{}\PYGZhy{} Chi\PYGZhy{}squared
   cosine            \PYGZhy{}\PYGZhy{} Cosine
   crystalball       \PYGZhy{}\PYGZhy{} Crystalball
   dgamma            \PYGZhy{}\PYGZhy{} Double Gamma
   dweibull          \PYGZhy{}\PYGZhy{} Double Weibull
   erlang            \PYGZhy{}\PYGZhy{} Erlang
   expon             \PYGZhy{}\PYGZhy{} Exponential
   exponnorm         \PYGZhy{}\PYGZhy{} Exponentially Modified Normal
   exponweib         \PYGZhy{}\PYGZhy{} Exponentiated Weibull
   exponpow          \PYGZhy{}\PYGZhy{} Exponential Power
   f                 \PYGZhy{}\PYGZhy{} F (Snecdor F)
   fatiguelife       \PYGZhy{}\PYGZhy{} Fatigue Life (Birnbaum\PYGZhy{}Saunders)
   fisk              \PYGZhy{}\PYGZhy{} Fisk
   foldcauchy        \PYGZhy{}\PYGZhy{} Folded Cauchy
   foldnorm          \PYGZhy{}\PYGZhy{} Folded Normal
   genlogistic       \PYGZhy{}\PYGZhy{} Generalized Logistic
   gennorm           \PYGZhy{}\PYGZhy{} Generalized normal
   genpareto         \PYGZhy{}\PYGZhy{} Generalized Pareto
   genexpon          \PYGZhy{}\PYGZhy{} Generalized Exponential
   genextreme        \PYGZhy{}\PYGZhy{} Generalized Extreme Value
   gausshyper        \PYGZhy{}\PYGZhy{} Gauss Hypergeometric
   gamma             \PYGZhy{}\PYGZhy{} Gamma
   gengamma          \PYGZhy{}\PYGZhy{} Generalized gamma
   genhalflogistic   \PYGZhy{}\PYGZhy{} Generalized Half Logistic
   genhyperbolic     \PYGZhy{}\PYGZhy{} Generalized Hyperbolic
   geninvgauss       \PYGZhy{}\PYGZhy{} Generalized Inverse Gaussian
   gibrat            \PYGZhy{}\PYGZhy{} Gibrat
   gompertz          \PYGZhy{}\PYGZhy{} Gompertz (Truncated Gumbel)
   gumbel\PYGZus{}r          \PYGZhy{}\PYGZhy{} Right Sided Gumbel, Log\PYGZhy{}Weibull, Fisher\PYGZhy{}Tippett, Extreme Value Type I
   gumbel\PYGZus{}l          \PYGZhy{}\PYGZhy{} Left Sided Gumbel, etc.
   halfcauchy        \PYGZhy{}\PYGZhy{} Half Cauchy
   halflogistic      \PYGZhy{}\PYGZhy{} Half Logistic
   halfnorm          \PYGZhy{}\PYGZhy{} Half Normal
   halfgennorm       \PYGZhy{}\PYGZhy{} Generalized Half Normal
   hypsecant         \PYGZhy{}\PYGZhy{} Hyperbolic Secant
   invgamma          \PYGZhy{}\PYGZhy{} Inverse Gamma
   invgauss          \PYGZhy{}\PYGZhy{} Inverse Gaussian
   invweibull        \PYGZhy{}\PYGZhy{} Inverse Weibull
   johnsonsb         \PYGZhy{}\PYGZhy{} Johnson SB
   johnsonsu         \PYGZhy{}\PYGZhy{} Johnson SU
   kappa4            \PYGZhy{}\PYGZhy{} Kappa 4 parameter
   kappa3            \PYGZhy{}\PYGZhy{} Kappa 3 parameter
   ksone             \PYGZhy{}\PYGZhy{} Distribution of Kolmogorov\PYGZhy{}Smirnov one\PYGZhy{}sided test statistic
   kstwo             \PYGZhy{}\PYGZhy{} Distribution of Kolmogorov\PYGZhy{}Smirnov two\PYGZhy{}sided test statistic
   kstwobign         \PYGZhy{}\PYGZhy{} Limiting Distribution of scaled Kolmogorov\PYGZhy{}Smirnov two\PYGZhy{}sided test statistic.
   laplace           \PYGZhy{}\PYGZhy{} Laplace
   laplace\PYGZus{}asymmetric    \PYGZhy{}\PYGZhy{} Asymmetric Laplace
   levy              \PYGZhy{}\PYGZhy{} Levy
   levy\PYGZus{}l
   levy\PYGZus{}stable
   logistic          \PYGZhy{}\PYGZhy{} Logistic
   loggamma          \PYGZhy{}\PYGZhy{} Log\PYGZhy{}Gamma
   loglaplace        \PYGZhy{}\PYGZhy{} Log\PYGZhy{}Laplace (Log Double Exponential)
   lognorm           \PYGZhy{}\PYGZhy{} Log\PYGZhy{}Normal
   loguniform        \PYGZhy{}\PYGZhy{} Log\PYGZhy{}Uniform
   lomax             \PYGZhy{}\PYGZhy{} Lomax (Pareto of the second kind)
   maxwell           \PYGZhy{}\PYGZhy{} Maxwell
   mielke            \PYGZhy{}\PYGZhy{} Mielke\PYGZsq{}s Beta\PYGZhy{}Kappa
   moyal             \PYGZhy{}\PYGZhy{} Moyal
   nakagami          \PYGZhy{}\PYGZhy{} Nakagami
   ncx2              \PYGZhy{}\PYGZhy{} Non\PYGZhy{}central chi\PYGZhy{}squared
   ncf               \PYGZhy{}\PYGZhy{} Non\PYGZhy{}central F
   nct               \PYGZhy{}\PYGZhy{} Non\PYGZhy{}central Student\PYGZsq{}s T
   norm              \PYGZhy{}\PYGZhy{} Normal (Gaussian)
   norminvgauss      \PYGZhy{}\PYGZhy{} Normal Inverse Gaussian
   pareto            \PYGZhy{}\PYGZhy{} Pareto
   pearson3          \PYGZhy{}\PYGZhy{} Pearson type III
   powerlaw          \PYGZhy{}\PYGZhy{} Power\PYGZhy{}function
   powerlognorm      \PYGZhy{}\PYGZhy{} Power log normal
   powernorm         \PYGZhy{}\PYGZhy{} Power normal
   rdist             \PYGZhy{}\PYGZhy{} R\PYGZhy{}distribution
   rayleigh          \PYGZhy{}\PYGZhy{} Rayleigh
   rice              \PYGZhy{}\PYGZhy{} Rice
   recipinvgauss     \PYGZhy{}\PYGZhy{} Reciprocal Inverse Gaussian
   semicircular      \PYGZhy{}\PYGZhy{} Semicircular
   skewcauchy        \PYGZhy{}\PYGZhy{} Skew Cauchy
   skewnorm          \PYGZhy{}\PYGZhy{} Skew normal
   studentized\PYGZus{}range    \PYGZhy{}\PYGZhy{} Studentized Range
   t                 \PYGZhy{}\PYGZhy{} Student\PYGZsq{}s T
   trapezoid         \PYGZhy{}\PYGZhy{} Trapezoidal
   triang            \PYGZhy{}\PYGZhy{} Triangular
   truncexpon        \PYGZhy{}\PYGZhy{} Truncated Exponential
   truncnorm         \PYGZhy{}\PYGZhy{} Truncated Normal
   truncweibull\PYGZus{}min  \PYGZhy{}\PYGZhy{} Truncated minimum Weibull distribution
   tukeylambda       \PYGZhy{}\PYGZhy{} Tukey\PYGZhy{}Lambda
   uniform           \PYGZhy{}\PYGZhy{} Uniform
   vonmises          \PYGZhy{}\PYGZhy{} Von\PYGZhy{}Mises (Circular)
   vonmises\PYGZus{}line     \PYGZhy{}\PYGZhy{} Von\PYGZhy{}Mises (Line)
   wald              \PYGZhy{}\PYGZhy{} Wald
   weibull\PYGZus{}min       \PYGZhy{}\PYGZhy{} Minimum Weibull (see Frechet)
   weibull\PYGZus{}max       \PYGZhy{}\PYGZhy{} Maximum Weibull (see Frechet)
   wrapcauchy        \PYGZhy{}\PYGZhy{} Wrapped Cauchy

Multivariate distributions
\PYGZhy{}\PYGZhy{}\PYGZhy{}\PYGZhy{}\PYGZhy{}\PYGZhy{}\PYGZhy{}\PYGZhy{}\PYGZhy{}\PYGZhy{}\PYGZhy{}\PYGZhy{}\PYGZhy{}\PYGZhy{}\PYGZhy{}\PYGZhy{}\PYGZhy{}\PYGZhy{}\PYGZhy{}\PYGZhy{}\PYGZhy{}\PYGZhy{}\PYGZhy{}\PYGZhy{}\PYGZhy{}\PYGZhy{}

.. autosummary::
   :toctree: generated/

   multivariate\PYGZus{}normal    \PYGZhy{}\PYGZhy{} Multivariate normal distribution
   matrix\PYGZus{}normal          \PYGZhy{}\PYGZhy{} Matrix normal distribution
   dirichlet              \PYGZhy{}\PYGZhy{} Dirichlet
   wishart                \PYGZhy{}\PYGZhy{} Wishart
   invwishart             \PYGZhy{}\PYGZhy{} Inverse Wishart
   multinomial            \PYGZhy{}\PYGZhy{} Multinomial distribution
   special\PYGZus{}ortho\PYGZus{}group    \PYGZhy{}\PYGZhy{} SO(N) group
   ortho\PYGZus{}group            \PYGZhy{}\PYGZhy{} O(N) group
   unitary\PYGZus{}group          \PYGZhy{}\PYGZhy{} U(N) group
   random\PYGZus{}correlation     \PYGZhy{}\PYGZhy{} random correlation matrices
   multivariate\PYGZus{}t         \PYGZhy{}\PYGZhy{} Multivariate t\PYGZhy{}distribution
   multivariate\PYGZus{}hypergeom \PYGZhy{}\PYGZhy{} Multivariate hypergeometric distribution

Discrete distributions
\PYGZhy{}\PYGZhy{}\PYGZhy{}\PYGZhy{}\PYGZhy{}\PYGZhy{}\PYGZhy{}\PYGZhy{}\PYGZhy{}\PYGZhy{}\PYGZhy{}\PYGZhy{}\PYGZhy{}\PYGZhy{}\PYGZhy{}\PYGZhy{}\PYGZhy{}\PYGZhy{}\PYGZhy{}\PYGZhy{}\PYGZhy{}\PYGZhy{}

.. autosummary::
   :toctree: generated/

   bernoulli                \PYGZhy{}\PYGZhy{} Bernoulli
   betabinom                \PYGZhy{}\PYGZhy{} Beta\PYGZhy{}Binomial
   binom                    \PYGZhy{}\PYGZhy{} Binomial
   boltzmann                \PYGZhy{}\PYGZhy{} Boltzmann (Truncated Discrete Exponential)
   dlaplace                 \PYGZhy{}\PYGZhy{} Discrete Laplacian
   geom                     \PYGZhy{}\PYGZhy{} Geometric
   hypergeom                \PYGZhy{}\PYGZhy{} Hypergeometric
   logser                   \PYGZhy{}\PYGZhy{} Logarithmic (Log\PYGZhy{}Series, Series)
   nbinom                   \PYGZhy{}\PYGZhy{} Negative Binomial
   nchypergeom\PYGZus{}fisher       \PYGZhy{}\PYGZhy{} Fisher\PYGZsq{}s Noncentral Hypergeometric
   nchypergeom\PYGZus{}wallenius    \PYGZhy{}\PYGZhy{} Wallenius\PYGZsq{}s Noncentral Hypergeometric
   nhypergeom               \PYGZhy{}\PYGZhy{} Negative Hypergeometric
   planck                   \PYGZhy{}\PYGZhy{} Planck (Discrete Exponential)
   poisson                  \PYGZhy{}\PYGZhy{} Poisson
   randint                  \PYGZhy{}\PYGZhy{} Discrete Uniform
   skellam                  \PYGZhy{}\PYGZhy{} Skellam
   yulesimon                \PYGZhy{}\PYGZhy{} Yule\PYGZhy{}Simon
   zipf                     \PYGZhy{}\PYGZhy{} Zipf (Zeta)
   zipfian                  \PYGZhy{}\PYGZhy{} Zipfian

An overview of statistical functions is given below.  Many of these functions
have a similar version in `scipy.stats.mstats` which work for masked arrays.

Summary statistics
==================

.. autosummary::
   :toctree: generated/

   describe          \PYGZhy{}\PYGZhy{} Descriptive statistics
   gmean             \PYGZhy{}\PYGZhy{} Geometric mean
   hmean             \PYGZhy{}\PYGZhy{} Harmonic mean
   pmean             \PYGZhy{}\PYGZhy{} Power mean
   kurtosis          \PYGZhy{}\PYGZhy{} Fisher or Pearson kurtosis
   mode              \PYGZhy{}\PYGZhy{} Modal value
   moment            \PYGZhy{}\PYGZhy{} Central moment
   skew              \PYGZhy{}\PYGZhy{} Skewness
   kstat             \PYGZhy{}\PYGZhy{}
   kstatvar          \PYGZhy{}\PYGZhy{}
   tmean             \PYGZhy{}\PYGZhy{} Truncated arithmetic mean
   tvar              \PYGZhy{}\PYGZhy{} Truncated variance
   tmin              \PYGZhy{}\PYGZhy{}
   tmax              \PYGZhy{}\PYGZhy{}
   tstd              \PYGZhy{}\PYGZhy{}
   tsem              \PYGZhy{}\PYGZhy{}
   variation         \PYGZhy{}\PYGZhy{} Coefficient of variation
   find\PYGZus{}repeats
   trim\PYGZus{}mean
   gstd              \PYGZhy{}\PYGZhy{} Geometric Standard Deviation
   iqr
   sem
   bayes\PYGZus{}mvs
   mvsdist
   entropy
   differential\PYGZus{}entropy
   median\PYGZus{}abs\PYGZus{}deviation

Frequency statistics
====================

.. autosummary::
   :toctree: generated/

   cumfreq
   percentileofscore
   scoreatpercentile
   relfreq

.. autosummary::
   :toctree: generated/

   binned\PYGZus{}statistic     \PYGZhy{}\PYGZhy{} Compute a binned statistic for a set of data.
   binned\PYGZus{}statistic\PYGZus{}2d  \PYGZhy{}\PYGZhy{} Compute a 2\PYGZhy{}D binned statistic for a set of data.
   binned\PYGZus{}statistic\PYGZus{}dd  \PYGZhy{}\PYGZhy{} Compute a d\PYGZhy{}D binned statistic for a set of data.

Correlation functions
=====================

.. autosummary::
   :toctree: generated/

   f\PYGZus{}oneway
   alexandergovern
   pearsonr
   spearmanr
   pointbiserialr
   kendalltau
   weightedtau
   somersd
   linregress
   siegelslopes
   theilslopes
   multiscale\PYGZus{}graphcorr

Statistical tests
=================

.. autosummary::
   :toctree: generated/

   ttest\PYGZus{}1samp
   ttest\PYGZus{}ind
   ttest\PYGZus{}ind\PYGZus{}from\PYGZus{}stats
   ttest\PYGZus{}rel
   chisquare
   cramervonmises
   cramervonmises\PYGZus{}2samp
   power\PYGZus{}divergence
   kstest
   ks\PYGZus{}1samp
   ks\PYGZus{}2samp
   epps\PYGZus{}singleton\PYGZus{}2samp
   mannwhitneyu
   tiecorrect
   rankdata
   ranksums
   wilcoxon
   kruskal
   friedmanchisquare
   brunnermunzel
   combine\PYGZus{}pvalues
   jarque\PYGZus{}bera
   page\PYGZus{}trend\PYGZus{}test
   tukey\PYGZus{}hsd

.. autosummary::
   :toctree: generated/

   ansari
   bartlett
   levene
   shapiro
   anderson
   anderson\PYGZus{}ksamp
   binom\PYGZus{}test
   binomtest
   fligner
   median\PYGZus{}test
   mood
   skewtest
   kurtosistest
   normaltest


Quasi\PYGZhy{}Monte Carlo
=================

.. toctree::
   :maxdepth: 4

   stats.qmc

Resampling Methods
==================

.. autosummary::
   :toctree: generated/

   bootstrap
   permutation\PYGZus{}test
   monte\PYGZus{}carlo\PYGZus{}test

Masked statistics functions
===========================

.. toctree::

   stats.mstats


Other statistical functionality
===============================

Transformations
\PYGZhy{}\PYGZhy{}\PYGZhy{}\PYGZhy{}\PYGZhy{}\PYGZhy{}\PYGZhy{}\PYGZhy{}\PYGZhy{}\PYGZhy{}\PYGZhy{}\PYGZhy{}\PYGZhy{}\PYGZhy{}\PYGZhy{}

.. autosummary::
   :toctree: generated/

   boxcox
   boxcox\PYGZus{}normmax
   boxcox\PYGZus{}llf
   yeojohnson
   yeojohnson\PYGZus{}normmax
   yeojohnson\PYGZus{}llf
   obrientransform
   sigmaclip
   trimboth
   trim1
   zmap
   zscore
   gzscore

Statistical distances
\PYGZhy{}\PYGZhy{}\PYGZhy{}\PYGZhy{}\PYGZhy{}\PYGZhy{}\PYGZhy{}\PYGZhy{}\PYGZhy{}\PYGZhy{}\PYGZhy{}\PYGZhy{}\PYGZhy{}\PYGZhy{}\PYGZhy{}\PYGZhy{}\PYGZhy{}\PYGZhy{}\PYGZhy{}\PYGZhy{}\PYGZhy{}

.. autosummary::
   :toctree: generated/

   wasserstein\PYGZus{}distance
   energy\PYGZus{}distance

Sampling
\PYGZhy{}\PYGZhy{}\PYGZhy{}\PYGZhy{}\PYGZhy{}\PYGZhy{}\PYGZhy{}\PYGZhy{}

.. toctree::
   :maxdepth: 4

   stats.sampling

Random variate generation / CDF Inversion
\PYGZhy{}\PYGZhy{}\PYGZhy{}\PYGZhy{}\PYGZhy{}\PYGZhy{}\PYGZhy{}\PYGZhy{}\PYGZhy{}\PYGZhy{}\PYGZhy{}\PYGZhy{}\PYGZhy{}\PYGZhy{}\PYGZhy{}\PYGZhy{}\PYGZhy{}\PYGZhy{}\PYGZhy{}\PYGZhy{}\PYGZhy{}\PYGZhy{}\PYGZhy{}\PYGZhy{}\PYGZhy{}\PYGZhy{}\PYGZhy{}\PYGZhy{}\PYGZhy{}\PYGZhy{}\PYGZhy{}\PYGZhy{}\PYGZhy{}\PYGZhy{}\PYGZhy{}\PYGZhy{}\PYGZhy{}\PYGZhy{}\PYGZhy{}\PYGZhy{}\PYGZhy{}

.. autosummary::
   :toctree: generated/

   rvs\PYGZus{}ratio\PYGZus{}uniforms

Distribution Fitting
\PYGZhy{}\PYGZhy{}\PYGZhy{}\PYGZhy{}\PYGZhy{}\PYGZhy{}\PYGZhy{}\PYGZhy{}\PYGZhy{}\PYGZhy{}\PYGZhy{}\PYGZhy{}\PYGZhy{}\PYGZhy{}\PYGZhy{}\PYGZhy{}\PYGZhy{}\PYGZhy{}\PYGZhy{}\PYGZhy{}

.. autosummary::
   :toctree: generated/

   fit

Circular statistical functions
\PYGZhy{}\PYGZhy{}\PYGZhy{}\PYGZhy{}\PYGZhy{}\PYGZhy{}\PYGZhy{}\PYGZhy{}\PYGZhy{}\PYGZhy{}\PYGZhy{}\PYGZhy{}\PYGZhy{}\PYGZhy{}\PYGZhy{}\PYGZhy{}\PYGZhy{}\PYGZhy{}\PYGZhy{}\PYGZhy{}\PYGZhy{}\PYGZhy{}\PYGZhy{}\PYGZhy{}\PYGZhy{}\PYGZhy{}\PYGZhy{}\PYGZhy{}\PYGZhy{}\PYGZhy{}

.. autosummary::
   :toctree: generated/

   circmean
   circvar
   circstd

Contingency table functions
\PYGZhy{}\PYGZhy{}\PYGZhy{}\PYGZhy{}\PYGZhy{}\PYGZhy{}\PYGZhy{}\PYGZhy{}\PYGZhy{}\PYGZhy{}\PYGZhy{}\PYGZhy{}\PYGZhy{}\PYGZhy{}\PYGZhy{}\PYGZhy{}\PYGZhy{}\PYGZhy{}\PYGZhy{}\PYGZhy{}\PYGZhy{}\PYGZhy{}\PYGZhy{}\PYGZhy{}\PYGZhy{}\PYGZhy{}\PYGZhy{}

.. autosummary::
   :toctree: generated/

   chi2\PYGZus{}contingency
   contingency.crosstab
   contingency.expected\PYGZus{}freq
   contingency.margins
   contingency.relative\PYGZus{}risk
   contingency.association
   fisher\PYGZus{}exact
   barnard\PYGZus{}exact
   boschloo\PYGZus{}exact

Plot\PYGZhy{}tests
\PYGZhy{}\PYGZhy{}\PYGZhy{}\PYGZhy{}\PYGZhy{}\PYGZhy{}\PYGZhy{}\PYGZhy{}\PYGZhy{}\PYGZhy{}

.. autosummary::
   :toctree: generated/

   ppcc\PYGZus{}max
   ppcc\PYGZus{}plot
   probplot
   boxcox\PYGZus{}normplot
   yeojohnson\PYGZus{}normplot

Univariate and multivariate kernel density estimation
\PYGZhy{}\PYGZhy{}\PYGZhy{}\PYGZhy{}\PYGZhy{}\PYGZhy{}\PYGZhy{}\PYGZhy{}\PYGZhy{}\PYGZhy{}\PYGZhy{}\PYGZhy{}\PYGZhy{}\PYGZhy{}\PYGZhy{}\PYGZhy{}\PYGZhy{}\PYGZhy{}\PYGZhy{}\PYGZhy{}\PYGZhy{}\PYGZhy{}\PYGZhy{}\PYGZhy{}\PYGZhy{}\PYGZhy{}\PYGZhy{}\PYGZhy{}\PYGZhy{}\PYGZhy{}\PYGZhy{}\PYGZhy{}\PYGZhy{}\PYGZhy{}\PYGZhy{}\PYGZhy{}\PYGZhy{}\PYGZhy{}\PYGZhy{}\PYGZhy{}\PYGZhy{}\PYGZhy{}\PYGZhy{}\PYGZhy{}\PYGZhy{}\PYGZhy{}\PYGZhy{}\PYGZhy{}\PYGZhy{}\PYGZhy{}\PYGZhy{}\PYGZhy{}\PYGZhy{}

.. autosummary::
   :toctree: generated/

   gaussian\PYGZus{}kde

Warnings / Errors used in :mod:`scipy.stats`
\PYGZhy{}\PYGZhy{}\PYGZhy{}\PYGZhy{}\PYGZhy{}\PYGZhy{}\PYGZhy{}\PYGZhy{}\PYGZhy{}\PYGZhy{}\PYGZhy{}\PYGZhy{}\PYGZhy{}\PYGZhy{}\PYGZhy{}\PYGZhy{}\PYGZhy{}\PYGZhy{}\PYGZhy{}\PYGZhy{}\PYGZhy{}\PYGZhy{}\PYGZhy{}\PYGZhy{}\PYGZhy{}\PYGZhy{}\PYGZhy{}\PYGZhy{}\PYGZhy{}\PYGZhy{}\PYGZhy{}\PYGZhy{}\PYGZhy{}\PYGZhy{}\PYGZhy{}\PYGZhy{}\PYGZhy{}\PYGZhy{}\PYGZhy{}\PYGZhy{}\PYGZhy{}\PYGZhy{}\PYGZhy{}\PYGZhy{}

.. autosummary::
   :toctree: generated/

   DegenerateDataWarning
   ConstantInputWarning
   NearConstantInputWarning
   FitError
\end{sphinxVerbatim}

\begin{sphinxVerbatim}[commandchars=\\\{\}]
/var/folders/p7/\PYGZus{}ryqngjd3jn\PYGZus{}ppndvnhdzqch0000gn/T/ipykernel\PYGZus{}33503/3286860601.py:2: DeprecationWarning: scipy.info is deprecated and will be removed in SciPy 2.0.0, use numpy.info instead
  sp.info(stats)
\end{sphinxVerbatim}

\end{sphinxuseclass}\end{sphinxVerbatimOutput}

\end{sphinxuseclass}
\begin{DUlineblock}{0em}
\item[] \sphinxstylestrong{\large Benchmarking}
\end{DUlineblock}

\sphinxAtStartPar
The \sphinxstylestrong{\%\%timeit} cell magic will execute the code in the cell and report the average time it took to execute it.

\begin{sphinxuseclass}{cell}\begin{sphinxVerbatimInput}

\begin{sphinxuseclass}{cell_input}
\begin{sphinxVerbatim}[commandchars=\\\{\}]
\PYG{n}{fruits} \PYG{o}{=} \PYG{p}{[}\PYG{l+s+s2}{\PYGZdq{}}\PYG{l+s+s2}{pomegranate}\PYG{l+s+s2}{\PYGZdq{}}\PYG{p}{,} \PYG{l+s+s2}{\PYGZdq{}}\PYG{l+s+s2}{cherry}\PYG{l+s+s2}{\PYGZdq{}}\PYG{p}{,} \PYG{l+s+s2}{\PYGZdq{}}\PYG{l+s+s2}{apricot}\PYG{l+s+s2}{\PYGZdq{}}\PYG{p}{,} \PYG{l+s+s2}{\PYGZdq{}}\PYG{l+s+s2}{date}\PYG{l+s+s2}{\PYGZdq{}}\PYG{p}{,} \PYG{l+s+s2}{\PYGZdq{}}\PYG{l+s+s2}{Apple}\PYG{l+s+s2}{\PYGZdq{}}\PYG{p}{,}
\PYG{l+s+s2}{\PYGZdq{}}\PYG{l+s+s2}{lemon}\PYG{l+s+s2}{\PYGZdq{}}\PYG{p}{,} \PYG{l+s+s2}{\PYGZdq{}}\PYG{l+s+s2}{Kiwi}\PYG{l+s+s2}{\PYGZdq{}}\PYG{p}{,} \PYG{l+s+s2}{\PYGZdq{}}\PYG{l+s+s2}{ORANGE}\PYG{l+s+s2}{\PYGZdq{}}\PYG{p}{,} \PYG{l+s+s2}{\PYGZdq{}}\PYG{l+s+s2}{lime}\PYG{l+s+s2}{\PYGZdq{}}\PYG{p}{,} \PYG{l+s+s2}{\PYGZdq{}}\PYG{l+s+s2}{Watermelon}\PYG{l+s+s2}{\PYGZdq{}}\PYG{p}{,} \PYG{l+s+s2}{\PYGZdq{}}\PYG{l+s+s2}{guava}\PYG{l+s+s2}{\PYGZdq{}}\PYG{p}{,}
\PYG{l+s+s2}{\PYGZdq{}}\PYG{l+s+s2}{Papaya}\PYG{l+s+s2}{\PYGZdq{}}\PYG{p}{,} \PYG{l+s+s2}{\PYGZdq{}}\PYG{l+s+s2}{FIG}\PYG{l+s+s2}{\PYGZdq{}}\PYG{p}{,} \PYG{l+s+s2}{\PYGZdq{}}\PYG{l+s+s2}{pear}\PYG{l+s+s2}{\PYGZdq{}}\PYG{p}{,} \PYG{l+s+s2}{\PYGZdq{}}\PYG{l+s+s2}{banana}\PYG{l+s+s2}{\PYGZdq{}}\PYG{p}{,} \PYG{l+s+s2}{\PYGZdq{}}\PYG{l+s+s2}{Tamarind}\PYG{l+s+s2}{\PYGZdq{}}\PYG{p}{,} \PYG{l+s+s2}{\PYGZdq{}}\PYG{l+s+s2}{Persimmon}\PYG{l+s+s2}{\PYGZdq{}}\PYG{p}{,}
\PYG{l+s+s2}{\PYGZdq{}}\PYG{l+s+s2}{elderberry}\PYG{l+s+s2}{\PYGZdq{}}\PYG{p}{,} \PYG{l+s+s2}{\PYGZdq{}}\PYG{l+s+s2}{peach}\PYG{l+s+s2}{\PYGZdq{}}\PYG{p}{,} \PYG{l+s+s2}{\PYGZdq{}}\PYG{l+s+s2}{BLUEberry}\PYG{l+s+s2}{\PYGZdq{}}\PYG{p}{,} \PYG{l+s+s2}{\PYGZdq{}}\PYG{l+s+s2}{lychee}\PYG{l+s+s2}{\PYGZdq{}}\PYG{p}{,} \PYG{l+s+s2}{\PYGZdq{}}\PYG{l+s+s2}{GRAPE}\PYG{l+s+s2}{\PYGZdq{}} \PYG{p}{]}
\end{sphinxVerbatim}

\end{sphinxuseclass}\end{sphinxVerbatimInput}

\end{sphinxuseclass}
\begin{DUlineblock}{0em}
\item[] \sphinxstylestrong{Benchmark with \sphinxstyleemphasis{for} loop}
\end{DUlineblock}

\begin{sphinxuseclass}{cell}\begin{sphinxVerbatimInput}

\begin{sphinxuseclass}{cell_input}
\begin{sphinxVerbatim}[commandchars=\\\{\}]
\PYG{o}{\PYGZpc{}\PYGZpc{}time}it 100
\PYG{n}{f1} \PYG{o}{=} \PYG{p}{[}\PYG{p}{]}
\PYG{k}{for} \PYG{n}{f} \PYG{o+ow}{in} \PYG{n}{fruits}\PYG{p}{:}
    \PYG{n}{f1}\PYG{o}{.}\PYG{n}{append}\PYG{p}{(}\PYG{n}{f}\PYG{p}{[}\PYG{p}{:}\PYG{l+m+mi}{3}\PYG{p}{]}\PYG{p}{)}
\end{sphinxVerbatim}

\end{sphinxuseclass}\end{sphinxVerbatimInput}
\begin{sphinxVerbatimOutput}

\begin{sphinxuseclass}{cell_output}
\begin{sphinxVerbatim}[commandchars=\\\{\}]
2.51 µs ± 23.4 ns per loop (mean ± std. dev. of 7 runs, 100000 loops each)
\end{sphinxVerbatim}

\end{sphinxuseclass}\end{sphinxVerbatimOutput}

\end{sphinxuseclass}
\begin{DUlineblock}{0em}
\item[] \sphinxstylestrong{\large Benchmark with list comprehension}
\end{DUlineblock}

\begin{sphinxuseclass}{cell}\begin{sphinxVerbatimInput}

\begin{sphinxuseclass}{cell_input}
\begin{sphinxVerbatim}[commandchars=\\\{\}]
\PYG{o}{\PYGZpc{}\PYGZpc{}time}it 100
\PYG{n}{f2} \PYG{o}{=} \PYG{p}{[}\PYG{n}{f}\PYG{p}{[}\PYG{p}{:}\PYG{l+m+mi}{3}\PYG{p}{]} \PYG{k}{for} \PYG{n}{f} \PYG{o+ow}{in} \PYG{n}{fruits}\PYG{p}{]}
\end{sphinxVerbatim}

\end{sphinxuseclass}\end{sphinxVerbatimInput}
\begin{sphinxVerbatimOutput}

\begin{sphinxuseclass}{cell_output}
\begin{sphinxVerbatim}[commandchars=\\\{\}]
3.23 µs ± 1.92 µs per loop (mean ± std. dev. of 7 runs, 100000 loops each)
\end{sphinxVerbatim}

\end{sphinxuseclass}\end{sphinxVerbatimOutput}

\end{sphinxuseclass}
\begin{DUlineblock}{0em}
\item[] \sphinxstylestrong{\large Images}
\end{DUlineblock}

\sphinxAtStartPar
You can insert images into doc cells using the Markdown image tag:

\sphinxAtStartPar
\sphinxstyleemphasis{The following uses a Markdown table to arrange the images.}


\begin{savenotes}\sphinxattablestart
\centering
\begin{tabulary}{\linewidth}[t]{|T|T|T|}
\hline
\sphinxstyletheadfamily 
\sphinxAtStartPar
Guido
&\sphinxstyletheadfamily 
\sphinxAtStartPar
Tim the Enchanter
&\sphinxstyletheadfamily 
\sphinxAtStartPar
Wombat
\\
\hline
\sphinxAtStartPar
\sphinxincludegraphics{{guido}.png}
&
\sphinxAtStartPar
\sphinxincludegraphics{{tim}.jpg}
&
\sphinxAtStartPar
\sphinxincludegraphics{{wombat}.jpg}
\\
\hline
\end{tabulary}
\par
\sphinxattableend\end{savenotes}

\begin{DUlineblock}{0em}
\item[] \sphinxstylestrong{\large The limerick}
\end{DUlineblock}

\sphinxAtStartPar
A dozen, a gross, and a score
Plus three times the square root of four
Divided by seven
Plus five times eleven
Is nine squared and not a bit more.







\renewcommand{\indexname}{Index}
\printindex
\end{document}